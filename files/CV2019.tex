%%%%%%%%%%%%%%%%%%%%%%%%%%%%%%%%%%%%%%%%%
% Medium Length Graduate Curriculum Vitae
% LaTeX Template
% Version 1.1 (9/12/12)
%
% This template has been downloaded from:
% http://www.LaTeXTemplates.com
%
% Original author:
% Rensselaer Polytechnic Institute (http://www.rpi.edu/dept/arc/training/latex/resumes/)
%
% Important note:
% This template requires the res.cls file to be in the same directory as the
% .tex file. The res.cls file provides the resume style used for structuring the
% document.
%
%%%%%%%%%%%%%%%%%%%%%%%%%%%%%%%%%%%%%%%%%

%----------------------------------------------------------------------------------------
%	PACKAGES AND OTHER DOCUMENT CONFIGURATIONS
%----------------------------------------------------------------------------------------

\documentclass[margin]{res} % Use the res.cls style, the font size can be changed to 11pt or 12pt here

\usepackage{helvet} % Default font is the helvetica postscript font
%\usepackage{newcent} % To change the default font to the new century schoolbook postscript font uncomment this line and comment the one above
\usepackage{enumerate}
\usepackage{url}
\usepackage{enumitem}
\usepackage{amsmath,amssymb}


 % Text width of the document
\setlength{\textwidth}{5.1in}
\setlength{\textheight}{260mm}

\newenvironment{list1}{
  \begin{list}{\ding{113}}{%
      \setlength{\itemsep}{0in}
      \setlength{\parsep}{0in} \setlength{\parskip}{0in}
      \setlength{\topsep}{0in} \setlength{\partopsep}{0in} 
      \setlength{\leftmargin}{0.17in}}}{\end{list}}
\newenvironment{list2}{
  \begin{list}{$\bullet$}{%
      \setlength{\itemsep}{0in}
      \setlength{\parsep}{0in} \setlength{\parskip}{0in}
      \setlength{\topsep}{0in} \setlength{\partopsep}{0in} 
      \setlength{\leftmargin}{0.2in}}}{\end{list}}



\begin{document}

%----------------------------------------------------------------------------------------
%	NAME AND ADDRESS SECTION
%----------------------------------------------------------------------------------------

%\name{{\Large Si\^an Zee Fryer} \\ Preferred name: Zee \\ Pronouns: they/them}
%\address{Department of Mathematics\\ University of California Santa Barbara \\ Santa Barbara, CA 93106}



\opening
\moveleft.5\hoffset\centerline{{\Huge \bf Zee Fryer}}% Your name at the top



\vspace{0.5em} 
\moveleft\hoffset\vbox{\hrule width\resumewidth height 0.7pt}\smallskip % Horizontal line after name; adjust line thickness by changing the '1pt'


% \moveleft.5\hoffset\centerline{Department of Mathematics}
% \moveleft.5\hoffset\centerline{University of California at Santa Barbara}
% \moveleft.5\hoffset\centerline{Santa Barbara, CA 93106}
% \vspace{1em}

% \moveleft.5\hoffset\centerline{sianfryer@math.ucsb.edu}
% \moveleft.5\hoffset\centerline{\url{http://www.math.ucsb.edu/~sianfryer/}}

%\vspace{1em}
%\moveleft\hoffset\vbox{\hrule width\resumewidth height 1pt}\smallskip % Horizontal line after name; adjust line thickness by changing the '1pt'
%----------------------------------------------------------------------------------------
\begin{resume}

% \moveleft\hoffset\vbox{\hrule width\resumewidth height 1pt}\smallskip % Horizontal line after name; adjust line thickness by changing the '1pt'

\section{\sc Contact}

\begin{tabular}{@{} p{5cm} p{6cm}}

{\em Pronouns}: they/them & {\em Location}: Bay Area, CA \\
%  &{\em Website}: \url{zeefryer.github.io} \\
{\em Email}: \url{fryer.zee@gmail.com} & {\em Website}: \url{zeefryer.github.io} \\
\end{tabular}



%----------------------------------------------------------------------------------------
%	OBJECTIVE SECTION
%----------------------------------------------------------------------------------------

\section{\sc Research Interests}  

\textbf{Quantum algebras}, in particular algebras of quantum matrices and their semi-classical limits, $H$-prime stratification, prime and primitive spectra.\\[0.3em]
\textbf{Totally nonnegative matrices}, including the combinatorics of total positivity in the real Grassmannian, Cauchon-Le diagrams, applications to noncommutative algebra, applications to quantum field theory.\\[0.3em]
\textbf{Related topics in algebra and combinatorics}, e.g. cluster algebra structures on the coordinate ring of the (quantum) Grassmannian, combinatorics of flag varieties.

%----------------------------------------------------------------------------------------
%	EDUCATION SECTION
%----------------------------------------------------------------------------------------

\section{\sc Academic Positions}
\textbf{University of California Santa Barbara}, 2016 - 2019 \\
Visiting Assistant Professor\\[0.3em]
\textbf{University of Leeds}, 2014 - 2016 \\
EPSRC Doctoral Prize Fellow

\section{\sc Education}

\textbf{University of Manchester}, 2010 - 2014 \\
PhD in Mathematics \\
Thesis: \textit{The $q$-Division Ring, Quantum Matrices and Semi-classical Limits} \\
% Advisor: Professor Toby Stafford \\[0.3em]
\textbf{University of Nottingham}, 2005 - 2009 \\
MMath in Pure Mathematics, 1st Class Hons. \\
Masters Dissertation: \textit{Involutions on Composition Algebras}\\
% Advisor: Dr Susanne Pumpl\"un 

 
\section{\sc Publications}
\begin{enumerate}[leftmargin=*]
\item S. Agarwala, {\bf S. Fryer}; A study in $\mathbb{G}_{\mathbb{R},\geq 0}$: from the geometric case book of Wilson loop diagrams and SYM $N = 4$. {\em Annals IHP D - Comb., Phys. and their Interactions (2021)}
\item S. Agarwala, {\bf S. Fryer}, K. Yeats; Combinatorics of the geometry of Wilson loop diagrams II: Grassmann necklaces, dimensions, and denominators. {\em Canadian Journal of Mathematics (2021)}
\item S. Agarwala, {\bf S. Fryer}, K. Yeats; Combinatorics of the geometry of Wilson loop diagrams I: equivalence classes via matroids and polytopes. {\em Canadian Journal of Mathematics (2021)}

\item S. Agarwala, {\bf S. Fryer}; An algorithm to construct the Le diagram associated to a Grassmann necklace. {\em Glasg. Math. J. (2019) 1-7}
\item {\bf S. Fryer}, T. Kanstrup, E. Kirkman, A. Shepler, S. Witherspoon; Color Lie Rings and PBW Deformations of Skew Group Algebras. {\em  J. Algebra 518 (2019), 211-236}
\item {\bf S. Fryer}, M. Yakimov; Separating Ore sets for Prime Ideals of Quantum Algebras. {\em Bull. Lond. Math. Soc. 49 (2017), no. 2, 202-215}
\item K. Casteels, {\bf S. Fryer}; From Grassmann necklaces to Restricted Permtuations and Back Again. {\em Algebr. Represent. Theory 20 (2017), no. 4, 895-921}
\item {\bf S. Fryer}; The Prime Spectrum of Quantum $SL_3$ and the Poisson-prime Spectrum of its Semi-classical Limit. {\em Trans. London Math. Soc. 4 (2017), no. 1, 1-29}
\item {\bf S. Fryer}; The $q$-Division Ring and its Fixed Rings. \textit{J. Algebra 402 (2014), 358-378} 

\end{enumerate}


	
\section{\sc Invited Talks}
AMS Fall Western Sectional Meeting 2018 in San Francisco CA, \textit{Special session on ``Homological Aspects of Noncommutative Algebra and Geometry'', October 27th 2018}\\[0.5em]
Joint Mathematics Meetings 2017 in Atlanta GA, \textit{Special session on ``New developments in noncommutative algebra and representation theory'', January 7th 2017}\\[0.5em]
AMS-EMS-SPM International Meeting in Porto, \textit{Special Session on ``Combinatorics and Geometry of Quantum Algebras'', June 13th 2015} \\[0.5em]
ARTIN meeting at University of Glasgow, \textit{November 7th 2014} \\[0.5em]
LMS Women in Mathematics Day, \textit{April 25th 2014} \\[0.5em]
ARTIN Young Researchers meeting at University of Newcastle, \textit{March 28th 2014} 



\section{\sc Teaching}

{\bf Calculus II}: Integration and applications (Fall 2017, Fall 2018, Spring 2019).\\[0.3em]
{\bf Calculus III}: Multivariable calculus (Spring 2017).\\[0.3em]
{\bf Differential Equations} (Winter 2017).\\[0.3em]
{\bf Transition to Higher Mathematics}: Introduction to pure mathematics and proofs (Fall 2016, Winter 2018, Fall 2018, Winter 2019).\\[0.3em]
{\bf Abstract Linear Algebra} (Spring 2018).\\[0.3em]
{\bf Abstract Algebra}: Fields and Galois Theory (Spring 2018).\\[0.3em]
{\bf Groups and Symmetry}: graduate level group theory (Fall 2015).


\section{\sc Outreach}

{\bf Contributed to the design, testing, and construction of Matt Parker's Domino Computer} \\
This project constructed a working 4-bit binary adder and an almost-working 5-bit binary adder entirely out of dominos at the Manchester Science Festival in 2012, as part of an outreach event to get children interested in how computers and logic gates work. See \url{https://www.youtube.com/watch?v=OpLU__bhu2w}.\\[-1.5em]

{\bf Volunteer at Manchester Girl Geeks outreach events}\\
Girl Geeks is an organization which encourages people of all genders and ages to learn about science, mathematics, and engineering topics in a safe and welcoming enviornment. Outreach events included a stall at Jodrell Bank's Science Arena encouraging young children to build and fly paper rockets, and a stall for International Women's Day highlighting Florence Nightingale's contribution to statistics.\\[-1.5em]

{\bf Presented posters explaining PhD research topic to a lay audience}\\
I presented a poster at WorldCon 2014 (a science-fiction convention) and at SET for Britain 2015, a competition for early career researchers to explain their research to Members of Parliament. The SET for Britain poster is available on my website.\\[-1.5em]

{\bf Math Circle instructor}\\
Math Circle is an organization which encourages children to learn through exploration and experimentation; the direction is set by the students, with the instructor acting mainly as a facilitator. In Spring 2018 I led a group of 7 year olds in a 10 week course exploring how fractions work.

\section{\sc Other}
Conference Organizer: ARTIN Young Researchers Meeting, March 6-7th 2015\\[0.5em]
Organising Committee: School of Mathematics Building Reopening Celebration 2015\\[0.5em]
Athena Swan Committee Member: School of Mathematics postdoc representative, January 2015 - January 2016 \\[0.5em]
School of Mathematics Postdoc Forum Organiser: December 2014 - June 2016


% \section{REFERENCES}
% \textbf{St\'ephane Launois} (University of Kent)\\
% \textbf{Robert Marsh} (University of Leeds)\\
% \textbf{Alison Parker} (University of Leeds, teaching reference)\\
% \textbf{Toby Stafford} (University of Manchester)









\end{resume}
\end{document}